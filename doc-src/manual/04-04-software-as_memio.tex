%%%%%%%%%%%%%%%%%%%%%%%%%%%%%%%%%%%%%%%%%%%%%%%%%%%%%%%%%%%%%%%%%%%%%%%%%%%%%%
%%
%% This file is part of the ASTERICS Framework. 
%%
%% Copyright (C) Hochschule Augsburg, University of Applied Sciences
%% Efficient Embedded Systems Group
%%
%% Author(s): Gundolf Kiefer <gundolf.kiefer@hs-augsburg.de>
%%
%%%%%%%%%%%%%%%%%%%%%%%%%%%%%%%%%%%%%%%%%%%%%%%%%%%%%%%%%%%%%%%%%%%%%%%%%%%%%%



\section{Transferring Data between Hardware and Software}\label{ch:04-software-data-trans}

\secauthor{Alexander Zöllner, Gundolf Kiefer}

\subsection{Brief Description}

The main purpose of the \asterics framework is enhancing image processing tasks using codesigns of hardware and software.
This requires transferring data between both subsystems in a convenient and efficient manner.
For accomplishing this task, the memory modules of \asterics are used for performing the actual data transfer. 
Since setting up data transfers has to be managed by the software stack, respective methods are provided.
These methods vary in their complexity and required overhead expected from the user.
The appropriate method can be chosen depending on the requirements of the application.

\subsection{Manual Data Transfer Management}
This method is the most direct way for transferring data by interfacing the corresponding memory module, using their module driver (see Chapter~\ref{ch:07-basic-mods-in_out}).
Here, a sufficient physically concurrent memory area has to be provided to the corresponding memory module, which is referred to as \textit{buffer} for (intermediately) storing data.
The user is supposed to manage the contents of the \textit{buffer} manually, to prevent its under- or overflows as well as inadvertently overwriting data.
Further, the status of the memory module has to be read from its hardware registers in order to determine whether a data transfer has been finished (\textit{state/control} register) or which parts of the \textit{buffer} have already been processed (\textit{current hw addr} register).
The double buffering scheme of the memory modules for queuing data transfers may be used for utilizing more than one \textit{buffer}.

Managing the data transfers directly gives the user exclusive control, allowing to also implement customized data transfer strategies.
However, preventing data loss has to be taken care of explicitly.


\subsection{POSIX-like Data Transfer Management}
For transferring data in a more convenient manner, the \asterics framework offers POSIX-like interfaces to the user, which are part of the software module \texttt{as\_memio} (memory input/output).
This module includes implementations for \textit{open}, \textit{read}, \textit{write} and \textit{close}, respectively.
Instead of having the user manage the data transfers and organizing the \textit{buffer(s)} for the memory module manually, these tasks are carried out by \textit{as\_memio}.
The user has to simply call \textit{open}, where the actual memory module is referenced.
The \texttt{as\_memio} module handles the internal specifics required for setting up data transfers.
These can be requested by calling either \textit{write} for transferring data to the hardware, using an \texttt{as\_memreader} module, or \textit{read} for obtaining data from the hardware processing chain, using an \texttt{as\_memwriter}.
Here, an opaque structure is provided to \texttt{as\_memio} (which is obtained by \textit{open}), along with the desired amount of data and a \textit{user buffer}.
This \textit{user buffer} is used for providing the data to be transferred or receiving data from \texttt{as\_memio}.
Once the request from the user has been served, the actual number of transferred bytes are returned to the user, which may be equal or less than the requested number.
Further, the \textit{user} buffer contains the amount of data, which has been served by the \texttt{as\_memio} module for \textit{read} calls.
For \textit{write} calls, the \textit{user buffer} is left untouched and may be used immediately by the user again after the function returns.
The \texttt{as\_memio} module internally manages the data for either direction and interfaces the associated memory module accordingly for transferring the data.
Provided the hardware processing chain supports the \texttt{STALL} mechanic (see Chapter~\ref{ch:05-03-interfaces-as_stream}), the \texttt{as\_memio} module can guarantee that no data is lost.




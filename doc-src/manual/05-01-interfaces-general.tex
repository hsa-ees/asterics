%%%%%%%%%%%%%%%%%%%%%%%%%%%%%%%%%%%%%%%%%%%%%%%%%%%%%%%%%%%%%%%%%%%%%%%%%%%%%%
%%
%% This file is part of the ASTERICS Framework. 
%%
%% Copyright (C) Hochschule Augsburg, University of Applied Sciences
%% Efficient Embedded Systems Group
%%
%% Author(s): Gundolf Kiefer <gundolf.kiefer@hs-augsburg.de>
%%            Philip Manke <philip.manke@hs-augsburg.de>
%%
%%%%%%%%%%%%%%%%%%%%%%%%%%%%%%%%%%%%%%%%%%%%%%%%%%%%%%%%%%%%%%%%%%%%%%%%%%%%%%



\section{General Interfaces} \label{ch:05-01-interfaces-general}

\secauthor{Alexander Zöllner, Gundolf Kiefer}


\subsection{Common Control and Status Registers} \label{ch:05-01-interfaces-general-common}

\infobox{Note: The following base registers are currently planned and will be integrated into the Automatics system generator in future version of \asterics.}

\begin{longtable}[ht]{|l|l|l|l|}
    \hline
    \multicolumn{1}{|c|}{\textbf{Name}} & \multicolumn{1}{c|}{\textbf{Address Off.}} & \multicolumn{1}{c|}{\textbf{Width}} & \multicolumn{1}{c|}{\textbf{Description}}\\
    \hline
    
    \texttt{asterics\_id} & 0x0 & 32 & \parbox{7cm}{\ \\
        Indicates whether the processing chain is \asterics-based.\\
    }\\
    \hline
    
    \texttt{asterics\_version} & 0x4 & 32 & \parbox{7cm}{\ \\
        Major, minor and revision of the \asterics installation this chain has been built with.\\
    }\\
    \hline
    
    \texttt{asterics\_driver\_id} & 0x8 & 32 & \parbox{7cm}{\ \\
        Compatible flavor of the software stack.\\
    }\\
    \hline
    
    \texttt{asterics\_state/control} & 0xC & 32 & \parbox{7cm}{\ \\
        Global \asterics-chain instructions and status information.\\
    }\\
    \hline
    
    \caption{Common control and status registers of an \asterics image processing chain.}
\end{longtable}

\begin{longtable}[ht]{|l|l|l|l|l|}
    \hline
    \multicolumn{1}{|c|}{\textbf{Field Name}} & \multicolumn{1}{c|}{\textbf{Bit Index}} & \multicolumn{1}{c|}{\textbf{Type}} & \multicolumn{1}{c|}{\textbf{Description}}\\
    \hline
    \texttt{asterics\_id} & 31:0 & ro & \parbox{5cm}{\ \\
        The identification field of \asterics (for example: 0x0ee500a5).\\
    }\\
    \hline
    \caption{Bit fields of the \texttt{asterics\_id} register of the \asterics-chain.}
    \label{table:common-control_asterics_id}
\end{longtable}

\begin{longtable}[ht]{|l|l|l|l|l|}
    \hline
    \multicolumn{1}{|c|}{\textbf{Field Name}} & \multicolumn{1}{c|}{\textbf{Bit Index}} & \multicolumn{1}{c|}{\textbf{Type}} & \multicolumn{1}{c|}{\textbf{Description}}\\
    \hline
    
    \texttt{major} & 31:16 & ro & \parbox{5cm}{\ \\
        Major number of the \asterics installation.\\
    }\\
    \hline
    
    \texttt{minor} & 15:8 & ro & \parbox{5cm}{\ \\
        Minor number of the \asterics installation.\\
    }\\
    \hline
    
    \texttt{revision} & 7:0 & ro & \parbox{5cm}{\ \\
        Revision of the \asterics installation.\\
    }\\
    \hline

    \caption{Bit fields of the \texttt{asterics\_version} register of the \asterics installation.}
    \label{table:common-control_asterics_version}
\end{longtable}

\begin{longtable}[ht]{|l|l|l|l|l|}
    \hline
    \multicolumn{1}{|c|}{\textbf{Field Name}} & \multicolumn{1}{c|}{\textbf{Bit Index}} & \multicolumn{1}{c|}{\textbf{Type}} & \multicolumn{1}{c|}{\textbf{Description}}\\
    \hline
    
    \texttt{driver\_id} & 31:0 & ro & \parbox{5cm}{\ \\
        Identification number of compatible software stack flavor.\\
        First 8 characters of the SHA1 hash of \texttt{as\_hardware.c}.\\
    }\\
    \hline
    
    \caption{Bit fields of the combined \texttt{asterics\_driver\_id} register of the \asterics-chain.}
    \label{table:common-control_asterics_driver_id}
\end{longtable}

\begin{longtable}[ht]{|l|l|l|l|l|}
    \hline
    \multicolumn{1}{|c|}{\textbf{Field Name}} & \multicolumn{1}{c|}{\textbf{Bit Index}} & \multicolumn{1}{c|}{\textbf{Type}} & \multicolumn{1}{c|}{\textbf{Reset Value}} & \multicolumn{1}{c|}{\textbf{Description}}\\
    \hline
    
    \texttt{reset} & 16 & wo & 0x0 & \parbox{5cm}{\ \\
        Reset the entire \asterics-chain at once.\\
    }\\
    \hline
    
    \texttt{sync\_error} & 3 & ro & 0x0 & \parbox{5cm}{\ \\
        Data synchronization error occurred anywhere within the \asterics-chain which led to data loss.\\
    }\\
    \hline
    
    \texttt{data\_error} & 2 & ro & 0x0 & \parbox{5cm}{\ \\
        A data error occurred anywhere within the \asterics-chain.\\
    }\\
    \hline
    
    \texttt{ready} & 0 & ro & n/a & \parbox{5cm}{\ \\
        \asterics-chain is ready for operation.\\
    }\\
    \hline
    
    \caption{Bit field overview of the combined \texttt{asterics\_state/control} register of the \asterics-chain.}
    \label{table:common-control_asterics_state_control}
\end{longtable}

The \texttt{READY} signal described in Table~\ref{table:common-control_asterics_state_control} is a "AND" combination of the \texttt{READY} signals of all hardware modules.
Similarly, the \texttt{DATA\_ERROR} and \texttt{SYNC\_ERROR} signal are "OR" combinations of their respective signals of all hardware modules.

Usually, the hardware modules also poses a \texttt{state/control} register with the same bit fields as the \texttt{asterics\_state/control} register.

\subsection{Error Handling}

There are two distinct error types available to \asterics, namely \texttt{SYNC\_ERROR} and \texttt{DATA\_ERROR}, which are handled separately.
The former is a per-module signal, as described in Table~\ref{05-02-module_signals} of Chapter~\ref{ch:05-02-interfaces-module_signals}, which may be forwarded to the \texttt{asterics\_state/control} register of the \asterics-chain.
The \texttt{SYNC\_ERROR} cannot be cleared by software directly. Rather, a reset has to be performed on the \asterics-chain (see Chapter~\ref{ch:05-02-interfaces-module_signals-error_handling}).
 
The \texttt{DATA\_ERROR} is part of the \texttt{as\_stream} interface (see Chapter~\ref{ch:05-03-interfaces-as_stream}).
As opposed to the \texttt{SYNC\_ERROR}, the software is responsible for clearing the \texttt{DATA\_ERROR}, as described in Chapter~\ref{ch:05-03-interfaces-as_stream-error_handling}.


\subsection{Module and Chain Reset Behavior}

Under certain circumstances, the need for reseting the \asterics image processing chain may arise, for example for having a defined state of the processing chain before using it for the first time or if an error occurred.

An \asterics-chain can either be reset at once using the \texttt{RESET} signal of the \texttt{asterics\_control} register or per-module.
The latter requires to perform the reset in the correct order, starting from the data source to sink.

If a module receives a \texttt{RESET} signal, the corresponding \texttt{READY} and \texttt{STROBE} signals have to be unset (0) after the following clock cycle.
\texttt{READY} and \texttt{STROBE} must not be set (1) as long as \texttt{RESET} is set (1).
The \texttt{RESET} signal has to persist for at least one clock cycle.
Once the \texttt{RESET} is unset (0), the module sets its \texttt{READY} signal as soon as data can be accepted or the \texttt{STALL} signal can be set correctly.

The software is expected to read the \texttt{READY} signal of all modules or the one of the \texttt{asterics\_state} register before starting any module.

\subsection{Version Management}

\asterics allows to build a wide variety of image processing chains, each being accompanied with their own \textit{\asterics Support Package}.
Thus, the \textit{\asterics Support Package} has to match the configuration of the \asterics-chain, which includes address offsets or the type of modules which have been used.
If the software does not match the hardware, unexpected behavior or the \asterics-chain not working at all is to be expected.
The most common error is software drivers trying to access the module's corresponding hardware registers at a certain address but inadvertently configure registers of a different type due to mismatch of addresses.

For this reason, the \asterics-chain poses version registers as described in Chapter~\ref{ch:05-01-interfaces-general-common}.
The software can read these registers to check whether the current version of the software stack is compatible with the \asterics-chain it is trying to operate.

The \textit{\asterics Support Package} also poses a version, which is a SHA1 hash of the source file \texttt{as\_hardware.c}.
This file describes the configuration of the \asterics-chain, along with pre-synthesis parameters of the used modules.



\subsection{\textit{i2c} Bus Master for Module Configuration}

\secauthor{Philip Manke}

%[[ TBD (PM): Prinzipielle Einführung, wie Module mit i2c-Slave-Schnittstelle %angebunden werden (Schnittstellen-Sicht); Details im Kapitel "Modules" (s.u.) ]]


External modules used with the \asterics framework (particularly cameras) often require to be configured using the \textit{i2c} bus. To provide a universal solution across platforms for this problem, \asterics implements its own \textit{i2c} hardware module.

This section will give a brief overview on what \textit{i2c} is and how to get started using the \texttt{as\_iic} module. 

\subsubsection{What is \textit{i2c}?}

\textit{i2c} is a bus system which allows for multiple devices to be connected using just two signals/wires in total. There are two types of devices which connect to the bus: masters and slaves.
Only masters are allowed to start a transaction on the bus. Having multiple masters on a single bus is not supported by the \texttt{as\_iic} module, though generally possible.

The two wires of the \textit{i2c} bus carry a clock signal (SCL \^= Serial CLock) and a data signal (SDA \^= Serial DAta).

The \textit{i2c} protocol uses device addresses to talk to the different devices on the bus. This address is 8 bits in length, with the last bit differentiating between read and write transactions.

\subsubsection{How to Configure \texttt{as\_iic} for Operation}

The \texttt{as\_iic} module only requires one function to be run before it is fully operational: \texttt{as\_iic\_init()}.
This configures the SCL frequency of the \textit{i2c} bus and is effective immediately.
The supported standard \textit{i2c} bus frequencies are 100 kHz and 400 kHz for \textit{standard mode} and \textit{fast mode i2c} respectively. With the \texttt{as\_iic} module it is possible to configure the frequency freely in a range from 10 kHz to 1 MHz.
This allows for faster transactions, provided the slave devices support the faster bus frequency. 
The function \texttt{as\_iic\_reset()} resets the module to the uninitialized state.

\subsubsection{Using \texttt{as\_iic}}

The software driver for \texttt{as\_iic} provides easy-to-use high level functions for executing transactions on the \textit{i2c} bus.
After connecting the hardware and initializing the \texttt{as\_iic} module correctly, the high level driver functions can be used immediately.

The two most useful functions are:

\begin{itemize}
    \item \texttt{as\_iic\_read\_reg()}
    \item \texttt{as\_iic\_write\_reg()}
\end{itemize}

These two functions will read and write the registers of \textit{i2c} slaves connected to the bus.
They both need the \textit{i2c} modules' base-address, the \textit{i2c} address of the slave and a pointer to the address of the register to read or write.
The forth pointer required by the functions points to the byte to send to the \textit{i2c} slave (\texttt{as\_iic\_write\_reg()}) or to the variable to store the content of the \textit{i2c} slave register in (\texttt{as\_iic\_read\_reg()}).

Section \ref{07-05-as-iic-connect-devices} briefly describes how to connect hardware devices to the \textit{i2c} bus. 

Details on the software driver for \texttt{as\_iic} and other modules, can be found in the \asterics Doxygen documentation.
The implementation of \texttt{as\_iic} is explained in detail in \ref{ch:07-05-modules-i2c}.

It is recommended to check out \ref{07-05-04-as-iic-quirks} before using \texttt{as\_iic}, to avoid common pitfalls.


\subsection{Generic Register Interface for \asterics Modules}
\label{sec:05-01-05-register_interface}
\secauthor{Philip Manke}

\asterics provides a generic and configurable register interface that all modules can use.
Configuring it becomes very easy when used in conjunction with Automatics, the \asterics chain generator.
Table \ref{table:05-01-slave_register_interface_signals} lists the ports comprising the slave register interface from the viewpoint of an ASTERICS module.

\begin{longtable}[ht]{|l|l|l|l|}
    \hline
    \multicolumn{1}{|c|}{\textbf{Name}} & \multicolumn{1}{c|}{\textbf{Data Type}} & \multicolumn{1}{c|}{\textbf{Direction}} & \multicolumn{1}{c|}{\textbf{Description}}\\
    \hline
    
    \texttt{slv\_status\_reg} & \texttt{slv\_reg\_data} & out & \parbox{6cm}{\ \\
        Data transport to software. An array of 32 bit registers.\\
    }\\
    \hline
    
    \texttt{slv\_ctrl\_reg} & \texttt{slv\_reg\_data} & in & \parbox{6cm}{\ \\
        Data transport to hardware. An array of 32 bit registers.\\
    }\\
    \hline
    
    \texttt{slv\_reg\_modify} & \texttt{std\_logic\_vector} & out & \parbox{6cm}{\ \\
        Data modify enable signals for the status registers. A vector as wide as the number of registers. Every clock cycle that a bit is set to \texttt{'1'} in this vector, the register will update it's value as set from the hardware.\\
    }\\
    \hline
    
    \texttt{slv\_reg\_config} & \texttt{slv\_reg\_config\_table} & out & \parbox{6cm}{\ \\
        A port to export the register configuration. An array of two bit \texttt{std\_logic\_vectors}, designating how the registers are configured.\\
    }\\
    \hline
    
    \caption{Overview of the signals comprising the slave register interface of \asterics modules.}
    \label{table:05-01-slave_register_interface_signals}
\end{longtable}

Each module using the interface must implement all four ports.
The \texttt{helpers} package from the \texttt{asterics} library must also be used in the VHDL file.
This makes the custom data types used to bundle the register signals available within the module.

Additionally, the module must contain a constant named \texttt{slave\_register\_configuration} towards the top of the architecture defintion:\\
\texttt{constant slave\_register\_configuration : slv\_reg\_config\_table(0 to <reg\_count>) := (<config>)}

The width of the constant and all ports must manually be set to the number of the desired register count.
Then the constant can be configured:
For each register a two bit wide \texttt{std\_logic\_vector} is required, table \ref{table:05-01-slave_register_config_values} lists all possible values.

\begin{longtable}[ht]{|c|c|}
    \hline
    \textbf{Value} & \textbf{Description}\\
    \hline
    
    \texttt{AS\_REG\_NONE} or \texttt{"00"} & \parbox{11cm}{\ \\
        No register generated.\\
    }\\
    \hline
    
    \texttt{AS\_REG\_STATUS} or \texttt{"01"} & \parbox{11cm}{\ \\
        Only generate a status register: Data transport only from the hardware module to the software. This saves some hardware resources and increases data security, as the software won't be able to overwrite the register contents.\\
    }\\
    \hline
    
    \texttt{AS\_REG\_CONTROL} or \texttt{"10"} & \parbox{11cm}{\ \\
        Only generate a control register: Data transport only from software towards the hardware module. This mainly just saves some hardware resources.\\
    }\\
    \hline
    
    \texttt{AS\_REG\_BOTH} or \texttt{"11"} & \parbox{11cm}{\ \\
        Generate a full slave register with data transport enabled in both directions.\\
    }\\
    \hline
    
    \caption{Possible values in the slave register configuration constant.}
    \label{table:05-01-slave_register_config_values}
\end{longtable}


\textbf{Example configuration:}\\
\texttt{constant slave\_register\_configuration : slv\_reg\_config\_table(0 to 3) := ("11", "11", "10", "01");}\\
For this configuration, all ports also need to have a data width of \texttt{(0 to 3)}. Also note that all register ports and the constant have data widths in an ascending direction (using \texttt{to} instead of \texttt{downto})!

\textbf{Note:} You may use an "others" construct to assign the register configuration (\texttt{:= ("11", (others => "01"))}. This is supported by Automatics for automatic IP-Core generation. \textit{Important:} This causes an error when using the constant definitions \texttt{AS\_REG\_XX} and may thus only be used with string literals (eg. \texttt{"01"})!

\textbf{Note:} If the module only has a single register, the "normal" assignment of the configuration constant using \texttt{:= ("10")} will usually fail. Either use an \texttt{others} construct or a direct assignment using \texttt{:= (0 => "10")} to assign the register configuration. Both ways are supported by Automatics.

\bigskip

\textbf{Using Automatics:}

Automatics can automatically instantiate the required register manager for the register interface.
This requires the following:
\begin{itemize}
\item The ports of the register interface must contain the names as mentioned above. They may have a \textit{common suffix}, which is required if a module contains multiple interfaces. The suffix must be separated by an underscore from the specified signal names and the configuration constant must have the same suffix. Note: The suffix is only required if the module has multiple register interfaces.
\item The configuration constant must be defined in the architecture definition, before any \texttt{begin} keyword, as Automatics stops parsing the architecture at the first \texttt{begin} keyword.
\end{itemize} 

\textbf{Manual configuration:}

The register interface ports must be connected to the register manager component \texttt{as\_regmgr}.
The register manager has an interface consisting of the same ports with identical names as the register interface of the hardware modules:
\begin{itemize}
\item \texttt{slv\_status\_reg => slv\_status\_reg}
\item \texttt{slv\_ctrl\_reg => slv\_ctrl\_reg}
\item \texttt{slv\_reg\_modify => slv\_reg\_modify}
\item \texttt{slv\_reg\_config => slv\_reg\_config}
\end{itemize}
\bigskip

And an interface roughly matching an AXI interface, ommiting unnecessary signals.
Table \ref{table:05-01-slave_register_axi_signals} lists these ports.
\begin{longtable}[ht]{|l|l|l|l|}
    \hline
    \multicolumn{1}{|c|}{\textbf{Name}} & \multicolumn{1}{c|}{\textbf{Data Type}} & \multicolumn{1}{c|}{\textbf{Direction}} & \multicolumn{1}{c|}{\textbf{Description}}\\
    \hline
    
    \texttt{sw\_address} & \texttt{std\_logic\_vector} & in & \parbox{6cm}{\ \\
        The read and write address. \texttt{as\_regmgr} does NOT support simultanious read and write accesses.\\
    }\\
    \hline
    
    \texttt{sw\_data\_out} & \texttt{std\_logic\_vector} & out & \parbox{6cm}{\ \\
        Data transport towards software.\\
    }\\
    \hline
    
    \texttt{sw\_data\_in} & \texttt{std\_logic\_vector} & in & \parbox{6cm}{\ \\
        Data transport towards hardware modules.\\
    }\\
    \hline
    
    \texttt{sw\_data\_out\_en} & \texttt{std\_logic} & in & \parbox{6cm}{\ \\
        Enable signal for read accesses from software.\\
    }\\
    \hline
    
    \texttt{sw\_data\_in\_en} & \texttt{std\_logic} & in & \parbox{6cm}{\ \\
        Enable signal for write accesses from software.\\
    }\\
    \hline
    
    \texttt{sw\_byte\_mask} & \texttt{std\_logic\_vector} & in & \parbox{6cm}{\ \\
        Byte-wise mask for partial write operations. Fully supported by \texttt{as\_regmgr}.\\
    }\\
    \hline
    
    \caption{Overview of the signals comprising the ports connected to the AXI Slave manager of the \asterics slave register manager.}
    \label{table:05-01-slave_register_axi_signals}
\end{longtable}


\textbf{Integration details:}

The software address is only as wide as the addressing of the hardware modules requires.
Therefore the required bits have to be extracted from the address provided by the AXI Slave manager.
The address will also have to be multiplexed from the read and write addresses.

The \texttt{sw\_data\_out} signal will also have to be multiplexed towards software (the AXI Slave manager).

Static HDL code has been developed and is used by Automatics to accomplish these tasks, shown in listing \ref{Code:05-01-06-regmgr_static}.
Note that the constants have to be assigned values chosen for the system that the code is used in.

\begin{lstlisting}[style=hdl, label=Code:05-01-06-regmgr_static, caption=Static code used to manage multiple \texttt{as\_regmgr} modules]
  -- Register interface constants and signals:
  -- Register address width in as_regmgr
  constant c_slave_reg_addr_width : integer := 6;
  -- Module addressing width: ceil(log2(module count))
  constant c_module_addr_width: integer := 2;
  -- Register addressing width: ceil(log2(register count per module))
  constant c_reg_addr_width : integer := 4;
  -- Number of as_regmgrs
  constant c_reg_if_count : integer := 4;
  -- Which module / as_regmgr is addressed?  
  signal read_module_addr : integer;
  -- Current address for as_regmgrs
  signal sw_address : std_logic_vector
                        (c_slave_reg_addr_width - 1 downto 0);
  -- Collect the sw_data_out signals of all as_regmgrs
  signal mod_read_data_arr : slv_reg_data(0 to c_reg_if_count - 1);

begin

  -- Extract the module address from the AXI read address
  read_module_addr <= to_integer(unsigned(
    axi_slv_reg_read_address(c_slave_reg_addr_width + 1 
                               downto c_reg_addr_width + 2)));

  -- Connect the read data out port
  --    of the register manager of the addressed module
  read_data_mux : process(mod_read_data_arr, read_module_addr, reset_n)
  begin
      if reset_n = '0' then
          axi_slv_reg_read_data <= (others => '0');
      else
          if read_module_addr < c_reg_if_count 
                and read_module_addr >= 0 then
              axi_slv_reg_read_data <= 
                  mod_read_data_arr(read_module_addr);
          else
              axi_slv_reg_read_data <= (others => '0');
          end if;
      end if;
  end process;

  -- Select between read and write address of the AXI interface
  --   depending on the read/write enable bits
  -- The register managers can only handle a single
  --   read/write per clock cycle
  -- Write requests have priority
  sw_addr_mux:
  process(axi_slv_reg_write_address, axi_slv_reg_read_address, 
            axi_slv_reg_write_enable, axi_slv_reg_read_enable)
  begin
    sw_address <= (others => '0');
    -- Disregarding lowest two bits
    --   to account for byte addressing on 32 bit registers
    if axi_slv_reg_write_enable = '1' then
        sw_address <= axi_slv_reg_write_address
                        (c_slave_reg_addr_width + 1 downto 2);
    elsif axi_slv_reg_read_enable = '1' then
        sw_address <= axi_slv_reg_read_address
                        (c_slave_reg_addr_width + 1 downto 2);
    else
        sw_address <= (others => '0');
    end if;
  end process;
\end{lstlisting}

\bigskip

The register managers have five Generics, configuring data widths and which address they should listen for, listed in table \ref{table:05-01-slave_register_manager_generics}.

\begin{longtable}[ht]{|l|l|l|l|}
    \hline
    \multicolumn{1}{|c|}{\textbf{Name}} & \multicolumn{1}{c|}{\textbf{Data Type}} & \multicolumn{1}{c|}{\textbf{Default Value}} & \multicolumn{1}{c|}{\textbf{Description}}\\
    \hline
    
    \texttt{REG\_ADDR\_WIDTH} & \texttt{integer} & 12 & \parbox{6cm}{\ \\
        The register address width: ceil(log2(number of register managers)) + ceil(log2(number of registers per module)).\\
    }\\
    \hline
    \texttt{REG\_DATA\_WIDTH} & \texttt{integer} & 32 & \parbox{6cm}{\ \\
        The register data width. Usually 32 bit.\\
    }\\
    \hline
    \texttt{MODULE\_ADDR\_WIDTH} & \texttt{integer} & 6 & \parbox{6cm}{\ \\
        Number of bits used to address \texttt{as\_regmgr} in this system.\\
    }\\
    \hline
    \texttt{REG\_COUNT} & \texttt{integer} & 32 & \parbox{6cm}{\ \\
        The number of registers for this \texttt{as\_regmgr}.\\
    }\\
    \hline
    \texttt{MODULE\_BASEADDR} & \texttt{integer} & None & \parbox{6cm}{\ \\
        The address for this \texttt{as\_regmgr}.\\
    }\\
    \hline
    
    \caption{Overview of Generics of the \asterics slave register manager.}
    \label{table:05-01-slave_register_manager_generics}
\end{longtable}

%%%%%%%%%%%%%%%%%%%%%%%%%%%%%%%%%%%%%%%%%%%%%%%%%%%%%%%%%%%%%%%%%%%%%%%%%%%%%%
%%
%% This file is part of the ASTERICS Framework. 
%%
%% Copyright (C) Hochschule Augsburg, University of Applied Sciences
%% Efficient Embedded Systems Group
%%
%% Author(s): Gundolf Kiefer <gundolf.kiefer@hs-augsburg.de>
%%
%%%%%%%%%%%%%%%%%%%%%%%%%%%%%%%%%%%%%%%%%%%%%%%%%%%%%%%%%%%%%%%%%%%%%%%%%%%%%%



\section{The \asterics 2D Window Filter Interface (\texttt{as\_window})} \label{ch:05-03-interfaces-as_window}

\secauthor{Philip Manke}

\subsection{Signal Overview}

The \texttt{as\_window} interface is used for modules requiring access to multiple pixels simultaneously from the image stream.
This is the case with filter modules, such as blurring, sharpening and edge detection filters and for operations such as non-maximum-suppression.
This interface provides the module with a rectangular window of pixels from the image stream, within the architecture of the 2D Window Pipeline, which is based on \cite{pohl_efficient_2014}.

The \texttt{as\_window} interface signals are listed in Table~\ref{05-04-as_2d_window_filter_signals}. Signals denoted with a right arrow ($\rightarrow$) in column "Direction" are outputs of the filter module. Accordingly, signals denoted with a left arrow ($\leftarrow$) are inputs for the filter module.
Optional signals are marked with an asterisk (*).

\begin{longtable}[ht]{|c|c|c|c|}
\hline 
\textbf{Signal} & \textbf{Direction} & \textbf{Data type} & \textbf{Description} \\
%% \small (* = optional) & \small (source <dir> sink) & \small (if optional) & \\%%
\hline 
\hline 
\endhead

\texttt{WINDOW<$x$><$y$><$b$>} & $\rightarrow$ & \texttt{t\_generic\_window} &  
\parbox{5,25cm}{ ~ \\ Data from 2D-sliding window buffer. \\ ~ \\ \small
The data type \texttt{t\_generic\_window} must be used for this interface and is defined in the \asterics VHDL library in the package \texttt{generic\_filter}.
The dimensions $x$ and $y$ describe the window width and height in pixels and $b$ describes the bit-width of the pixels.
A filter module may have more than one \texttt{window\_in} port, though they must both be differentiated using suffixes and/or prefixes.
\\ ~ } \\

\hline 

\texttt{STROBE} & $\rightarrow$ & \texttt{std\_logic} &
\parbox{5,25cm}{ ~ \\ A new valid data item is present at the \texttt{WINDOW} port. \\ ~ \\ \small
\\ ~ } \\

\hline 

\texttt{STALL (*)} & $\leftarrow$ & \texttt{std\_logic} &
\parbox{5,25cm}{ ~ \\ Request to pause sending data. For details, see Section \ref{05-03-stall}.
\\ ~ } \\

\hline

\caption{Signals of a \texttt{2D window filter} interface} \label{05-04-as_2d_window_filter_signals}\\
\end{longtable}

%%%%%%%%%%%%%%%%%%%%%%%%%%%%%%%%%%%%%%%%%%%%%%%%%%%%%%%%%%%%%%%%%%%%%%%%%%%%%%
%%
%% This file is part of the ASTERICS Framework. 
%%
%% Copyright (C) Hochschule Augsburg, University of Applied Sciences
%% Efficient Embedded Systems Group
%%
%% Author(s): Gundolf Kiefer <gundolf.kiefer@hs-augsburg.de>
%%
%%%%%%%%%%%%%%%%%%%%%%%%%%%%%%%%%%%%%%%%%%%%%%%%%%%%%%%%%%%%%%%%%%%%%%%%%%%%%%



\section{2D Pipeline Generator} \label{ch:06-04-tools-pipegen}

\secauthor{Alexander Zöllner}

\subsection{Brief Description}
The \textit{2D Pipeline Generator} is a script-based tool for building a 2D pipeline in a semi-automatic way.
Required filter masks for the 2D window modules are extracted from images using the program Gimp.
The resulting 2D pipeline consists of storage elements for each data type and a number of \textit{2D Window Interfaces}.

Note: A similar functionality has been implemented into the Python script based generator Automatics, which will eventually supersede this generator tool.

\subsubsection{Hardware Components}

Table~\ref{06-04-pipeline-generator-components} lists the required hardware components for the 2D Pipeline Generator in order to build the pipeline.

\begin{longtable}[ht]{|c|c|}
\hline 
\textbf{Component} & \textbf{Purpose}\\
\hline 
\hline 
\endhead

\texttt{myLittleHelpersPkg.vhd} & 
\parbox{7cm}{ ~ \\ 
Defines several mathematic functions such as finding the exponent of two required for representing a given integer.
\\ ~ } \\

\hline 

\texttt{dram\_shift\_reg.vhd} & 
\parbox{7cm}{ ~ \\ 
Implements the logic required to store data in the 2D-sliding window buffer using distributed RAM (DRAM), i.e. slice registers. DRAM is used for small sections up to 100 elements.
\\ ~ } \\

\hline 

\texttt{bram\_shift\_reg.vhd} & 
\parbox{7cm}{~ \\ 
Implements the logic required to store data in the 2D-sliding window buffer using Block-RAM (BRAM). BRAM is used for long sections which require more than 100 elements.
\\ ~ } \\

\hline 

\texttt{myBRAM.vhd} & 
\parbox{7cm}{~ \\ 
Implements the actual storage element in BRAM.
\\ ~ } \\

\hline 

\caption{Hardware components required for the 2D Pipeline Generator} \label{06-04-pipeline-generator-components}\\
\end{longtable}

\subsubsection{Generator Input}
In order to build the 2D pipeline, the 2D Pipeline Generator has to determine the required interfaces.
Therefore, an image for each filter mask has to be provided using the image program Gimp (of version 2.8).
In the first step, a new image has to be created. 
The height of the image has to match the one of the filter mask and the width the resolution of the image in x direction.
For the filter mask, a new layer has to be created, matching the size of the image.
The layer should be named appropriately, since the generator uses the same name for the corresponding interface of the filter mask.
Further, the background layer has to be deleted.
The in- and output positions of the 2D window module are chosen by coloring the associated pixel, as shown in table~\ref{06-04-pipeline-generator-filter-masks}.
Reference pixels are solely for visual user assistance, e.g. marking the center pixel of a Gaussian filter mask.
In a final step, the generated image has to be exported in a png format.

\begin{longtable}[ht]{|c|c|c|c|}
  \hline
  & In & Ref. & Out \\
  \hline
  In & \cellcolor{red!75}0xff0000 & \cellcolor{yellow!75}0xffff00 & \cellcolor{magenta!75}0xff00ff \\
  \hline
  Ref. & \cellcolor{yellow!75}0xffff00 & \cellcolor{green!75}0x00ff00 & \cellcolor{cyan!25}0x00ffff \\
  \hline
  Out & \cellcolor{magenta!75}0xff00ff & \cellcolor{cyan!25}0x00ffff & \cellcolor{blue!75}0x0000ff \\
  \hline
  \caption{Color encoding used for I/O configuration} \label{06-04-pipeline-generator-filter-masks}
\end{longtable}

The in- and output ports of the filter masks have to be associated with their corresponding data type to infer the required amount of storage elements and connecting them correctly.
For this reason, an input file, namely \textit{data.txt} has to be provided, which contains labels for all data types with their associated bit width.
An entry has the format \texttt{<label>:<bit-width>}, where the \textit{label} must only contain letters (a-z, A-Z) and the \textit{bit-width} only digits (0-9).

Data types are associated with the filter masks by assigning the labels to the ports of a given filter port.
This is done in the \textit{connections.txt}, which is an output of the 2D Pipeline Generator.


\subsubsection{Scripts}

\begin{longtable}[ht]{|c|c|}
\hline 
\textbf{Name} & \textbf{Purpose} \\
\hline 
\hline 
\endhead

\texttt{generateConnectionFile.py} & 
\parbox{7cm}{ ~ \\ 
Generates the file connections.txt from the png files containing the filter masks and data.txt for the data types.
\\ ~ } \\

\hline 

\texttt{generateDataPipeline.py} & 
\parbox{7cm}{ ~ \\ 
Generates the actual 2D pipeline.
\\ ~ } \\

\hline 

\texttt{dataPipeline.py} & 
\parbox{7cm}{ ~ \\ 
Implements the functionality of the 2D Pipeline Generator and provides methods for generateConnectionFile.py and generateDataPipeline.py
\\ ~ } \\

\hline 

\caption{Scripts of the 2D Pipeline Generator} \label{06-04-pipeline-generator-scripts}\\
\end{longtable}

\subsection{Output Products}

\begin{longtable}[ht]{|c|c|}
\hline 
\textbf{Name} & \textbf{Purpose}\\
\hline 
\hline 
\endhead

\texttt{connections.txt} & 
\parbox{7cm}{ ~ \\ 
Lists ports of all filter masks. Has to be edited by user.
\\ ~ } \\

\hline 

\texttt{dataLayers.log} & 
\parbox{7cm}{ ~ \\ 
Lists the BRAM resource consumption as well as implementation details for individual layers.
\\ ~ } \\

\hline 

\texttt{dataPipeline.vhd} & 
\parbox{7cm}{~ \\ 
The vhdl module containing the 2D pipeline to be instantiated by the user.
\\ ~ } \\

\hline 

\texttt{dataPipeline\_mapping.vhd} & 
\parbox{7cm}{~ \\ 
Provides signal declarations and component instantiation in vhdl, which can be copied in the user design.
\\ ~ } \\

\hline 

\caption{Output products of the 2D Pipeline Generator} \label{06-04-pipeline-generator-output}\\
\end{longtable}